\documentclass[a4paper, 12pt]{article}
\usepackage{amsmath, amssymb}
\newcommand{\Mod}[1]{\ \mathrm{mod}\ #1}

\title{COMPSCI 120 Assignment One}
\author{Alexander Bailey}

\begin{document}
\maketitle
\section*{Question 1}
\subsection*{a.}
\subsection*{b.}
\begin{center}
    \textit{Find $152615278636986567767^{12345678} \Mod 5$}
\end{center}
\begin{align*}
    152615278636986567767^{12345678} \Mod 10 = 7^{12345678} \\
    (7^2) \Mod 10 = 49 \Mod 10 = 9 \Mod 10 \\
    (7^4) \Mod 10 = (7^2 \cdot 7^2) \Mod 10 = (9 \cdot 9) \Mod 10 = 1 \Mod 10 \\
    12345678 = 308669 \cdot 4 + 2 \\
    (7^{4k} \cdot 7^2) \Mod 10 = (1\cdot9)\Mod10 = 9 \\
    9 \Mod 5 = 4 \\
    152615278636986567767^{12345678} \Mod 5 = 4 \\
\end{align*} 

\subsection*{c.}
\section*{Question 2}
\subsection*{a.}
\subsection*{b.}
\subsection*{c.}
\subsection*{d.}
\section*{Question 3}
\subsection*{a.}
\begin{center}
    \textit{Check whenever 1928467 is a UPC} \\
\end{center}

There is only one condition for n to be a UPC.
\begin{align*}
    c = (3M + N)\%10
\end{align*}
Where:
\begin{itemize}
    \item The sum of the odd position digits (not including the last) is M 
    \item The sum of the even position digits (not including the last) is N 
\end{itemize}
So for 1928467:
\begin{align*}
    M = 1 + 2 + 4 = 7 \\
    N = 9 + 8 + 6 = 23  \\
    \therefore C = (3\cdot7 + 23)\%10 = 4 \\
    C \neq 0 \\
    7 \neq 10 - 4 \\
\end{align*}
$\therefore$ 1928467 is not a UPC 

\subsection*{b.}
\subsection*{c.}
\section*{Question 4}
\subsection*{a.}
\subsection*{b.}
\end{document}
